\documentclass[conference]{IEEEtran}

\IEEEoverridecommandlockouts

\usepackage[nomain, acronym, symbols]{glossaries}
\usepackage[english]{babel}
\usepackage[numbers]{natbib}
\usepackage{hyperref}

\loadglsentries{acronyms}

% TODO: Authors
% TODO: Introduction
% TODO: Basic concepts
% TODO: Related works
% TODO: Developement
% TODO: Results and discussion
% TODO: Conclusion

\begin{document}

\title{Fine-tuning of LLaMA for Skin Lesion Classification and Medical Report Generation}

\author{\IEEEauthorblockN{1\textsuperscript{st} Eric Fernandes Evaristo}
\IEEEauthorblockA{\textit{dept. name of organization (of Aff.)} \\
\textit{name of organization (of Aff.)}\\
City, Country \\
email address or ORCID}
\and
\IEEEauthorblockN{2\textsuperscript{nd} Aldo von Wangenheim}
\IEEEauthorblockA{\textit{dept. name of organization (of Aff.)} \\
\textit{name of organization (of Aff.)}\\
City, Country \\
email address or ORCID}
\and
\IEEEauthorblockN{3\textsuperscript{rd} Rodrigo de Paula e Silva Ribeiro}
\IEEEauthorblockA{\textit{dept. name of organization (of Aff.)} \\
\textit{name of organization (of Aff.)}\\
City, Country \\
email address or ORCID}
\and
}

\maketitle

\begin{abstract}
    The early detection of skin lesions is crucial for the treatment of severe diseases, such as skin cancer. However, these lesions are usually only diagnosed by dermatologists. Given that these professionals are not always available in regions with less healthcare infrastructure, it would be useful to have a tool capable of skin lesion classification and medical report generation. For this purpose, we fine-tuned \gls{llama} 3.2 11B with \gls{qlora} and \gls{lora}. The fine-tuning was performed in two scenarios, in the first scenario, we trained the model with the \gls{ham10000} dataset for classification tasks only, achieving an accuracy of up to 86.2\% with \gls{lora}. In the second scenario, the model was trained with the \gls{sttsc} dataset for both classification and report generation. The final models presented a classification accuracy of 45.2\% with \gls{qlora} and 44.5\% with \gls{lora}. This low performance can be explained by inconsistencies in the \gls{sttsc} dataset.
\end{abstract}

\begin{IEEEkeywords}
    % Some keywords won't need acronyms anyway
    Skin Lesions, Skin Cancer, MLLM, \gls{llama}, Fine-tuning, \gls{qlora}, \gls{lora}.
\end{IEEEkeywords}

\section{Introduction}

Several skin diseases can have symptons such as skin lesions, most notably, skin cancer, which is specially relevant. In Brazil, 30\% of malignant tumors are attributed to skin cancer \cite{skin_cancer_in_brazil}. An accurate and early classification of the lesion category is fundamental for an adequate treatment. However, this is a complex task that depends on many visual aspects, requiring then the analysis from trained professionals, such as dermatologists \cite{habif2015clinical, skin_cancer_survival}.

The early detection of skin cancer has a socio-economic factor. As noted in studies by \citet{skin_cancer_socioeconomic} and \citet{santos2020desigualdades}, those who present tumors in more advanced stages tend come from a worse socio-economic background. This can be correlated with the lack of access to healthcare and dermatologists for these patients. However, in the brazilian context, healthcare is partially provided by \gls{acs}, or Community Health Agents in english, which can offer basic health assistance and are also in contact with poorer communities \cite{filgueiras2011agente}.

In this context, a tool with ease of use and capable of classifying skin lesions and generating reports would be useful, as it could be utilized by \gls{acs} as form of screening. Thus, addressing the lack of dermatologists for part of the population. The requirements for this tool can be addressed adequately by a \gls{mllm}. Therefore, for in this work, \gls{llama}-3.2-11B was chosen % TODO: Continue

\bibliographystyle{IEEEtranN}
\bibliography{references}

\end{document}

